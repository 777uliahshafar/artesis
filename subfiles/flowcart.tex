\documentclass[../main.tex]{subfiles}
\begin{document}
\section{Flowchart Example TiKz}

\lipsum[1-5]
%\begin{figure}[hp]
%\centering
%\begin{tikzpicture}[node distance=2cm]
%\node (ltr) [startstop] {Latar Belakang};
%
%\node (rum) [startstop, right of=ltr, xshift=2cm] {Perumusan Masalah};
%
%\node (tuj) [startstop, below of=rum, yshift=0.5cm] {Tujuan Penelitian};
%
%
%\node (pus) [startstop, below of=tuj, yshift=0.5cm] {Studi Pustaka};
%
%
%\node (kaj) [startstop, below of=pus, text width=3.5cm, xshift= -4cm, yshift=.5cm] {
%	\textbf{Kajian Teori}\\ - Fitur binaan\\ - Aktivitas Luar
%};
%
%
%\node (kaj2) [startstop, below of=pus, text width=3.5cm, xshift= 4cm, yshift=.5cm] {
%	\textbf{Gambaran Objek}\\ Fitur Binaan dan Aktivitas Luar Jl. Pinggir Laut
%};
%
%
%\node (hip) [startstop, below of=pus, yshift=-.5cm] {Hipotesa};
%
%
%\node (met) [startstop, below of=hip, yshift=-.75cm, text width=7cm] {
%	\textbf{Metode Peneltian}\\ Menggunakan Metode penelitian Kuantitatif Rasionalistik
%
%	\textbf{Variabel}\\
%	- Bebas : Fitur Binaan\\
%	- Terikat : Aktivitas Luar\\
%
%	\textbf{Sumber data}: Observasi dan Kuesioner
%};
%
%\node (ana) [startstop, below of=met, text width=8cm, yshift=-2cm] {
%		\textbf{Analisis Data Statistik}\\ Penelitian ini menggunakan metode statika berupa uji regresi guna mengetahui pengaruh variabel fitur binaan terhadap variabel aktivitas luar.
%};
%
%\node (tem) [startstop, below of=ana, yshift=-.25cm] {Temuan Penelitian};
%
%\node (kes) [startstop, below of=tem, yshift=.6cm] {Kesimpulan dan Rekomendasi};
%
%\draw [arrow] (ltr) -- (rum);
%\draw [arrow] (rum) -- (tuj);
%\draw [arrow] (tuj) -- (pus);
%
%\draw [arrow] (pus) -| (kaj);
%\draw [arrow] (pus) -| (kaj2);
%
%\draw [doublearrow] (kaj) -- (kaj2);
%
%\draw [arrow] (kaj) |- (met);
%\draw [dotted] (kaj) |- (hip);
%
%\draw [arrow] (kaj2) |- (met);
%\draw [dotted] (kaj2) |- (hip);
%
%\draw [arrow] (met) -- (ana);
%\draw [arrow] (ana) -- (tem);
%
%\draw [arrow] (tem) -- (kes);
%
%\end{tikzpicture}
%\caption{Alur Pikir}
%\end{figure}


\section{Flowchart Example 2 TiKz}

\lipsum[6-10]
\begin{figure}[htbp]
\centering
\begin{tikzpicture}[node distance=2cm]

	\node (tit) [startstop, text width= 5cm] {Fitur Fisik Binaan pada Aktivitas Luar Jl. Pinggir Laut};

	\node (va1) [startstop, below of=tit, text width=5cm, xshift=-3cm] {Variabel Bebas\\ Fitur Fisik Binaan};

	\node (va2) [startstop, below of=tit, text width=5cm, xshift=3cm] {Variabel Tergantung\\ Aktivitas Luar};

	\node (de1) [startstop, below of=va1, text width=5cm, yshift=-2cm] {
		\textbf{Sub Variabel Bebas}\\
		- Elemen Jalan \\
		- Kualitas Jalan \\
		- Elemen Tempat Duduk \\
		- Kualitas Tempat Duduk \\
		- Elemen Alami \\
		- Kualitas Alami \\
		- Fasilitas \& Aminities \\
		- Estetika \\

	};
	\node (de2) [startstop, below of=va2, text width=5cm, yshift=-2cm] {
			\textbf{Sub Variable Tergantung}\\
		- Aktivitas relaxsasi\\
		- Aktivitas fisik\\
		- Travel aktif\\
		- Interaction with wildlife and nature\\
		- Interaksi sosial\\
		- Partisipasi di aktivitas grup\\
		};
\draw [arrow] (tit) -| (va1);
\draw [arrow] (va1) -- (de1);
\draw [arrow] (tit) -| (va2);
\draw [arrow] (va2) -- (de2);

\end{tikzpicture}
\caption{Alur Pikir}
\end{figure}

\section{readme}
\lipsum[11-12]
% Comment tikz flowchart with command :x,xxs/^/%
% Uncomment tikz flowchart with command :x,xxs/^%/
% xx = linenumbers


%\onlyinsubfile{\biblio}
\end{document}

